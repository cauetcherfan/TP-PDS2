Sistema parcial de biblioteca.

O programa implementado visa demonstrar, na prática, conceitos aprendidos relacionados à orientação a objetos. Foram implementadas três classes, sendo elas\+:
\begin{DoxyItemize}
\item Classe Usuario (Classe-\/\+Mãe);
\item Classe Cliente (Subclasse);
\item Classe Bibliotecario (Subclasse);
\end{DoxyItemize}

Com tais classes em mão, implementamos uma relação de herança entre a Classe-\/\+Mãe e suas respectivas subclasses. Tal herança é observada na própria declaração das subclasses, onde utilizamos \char`\"{}\+Class (nome da subclasse) \+: public Usuario\{...\}\char`\"{}. O uso de polimorfismo também pode ser observado, de modo prático, na função exibe\+Menu(), a qual é apenas uma interface na Classe-\/\+Mãe (Usuario) e, nas subclasses, é implementado um comportamento específico para cada Classe-\/\+Filha, realizando assim o conceito de polimorfismo aprendido ao longo do curso. Como solicitado, o programa foi criado de maneira suscinta e objetiva, onde apenas algumas funções necessárias para o funcionamento básico foram implementadas, como por exemplo as funcionalidades \char`\"{}logoff\char`\"{} e \char`\"{}sair\char`\"{} que, quando acionadas, realizam o comportamento esperado.

Para compilar o programa, basta o utilizar o comando make no terminal e, para rodá-\/lo, basta digitar ./main.out Uma vez compilado e pronto para ser executado, deve-\/se usar o nome \char`\"{}\+Matheus\char`\"{} e a senha \char`\"{}123\char`\"{} para logar como bibliotecário no sistema e, para logar como cliente, deve-\/se usar o nome \char`\"{}\+Cauet\char`\"{} e a respectiva senha \char`\"{}321\char`\"{}, dessa forma você poderá visualizar e testar todas as funcionalidades implementadas no projeto. 